\documentclass{article}
\usepackage{hyperref}


\begin{document}
    
Title: Rapid Deployment of Curved Surfaces via Programmable Auxetics

Link: \url{https://lgg.epfl.ch/publications/2018/ProgrammableAuxetics/paper.pdf} 

\section*{Contributions}
The deployable range of an auxetic structure can be controlled by the size of unit cells
as well as how "deployed" the unit cells are in the 2D configuration. This allows for 
control of the deployed curve surface because nonuniform forces will lead to buckling. 

Actuation done via inflation or by gravity. These types of actuation allow for maximum
deployment which ensures that the deployed structure matches the target shape as close 
as possible. 

\subsection*{Inflation Actuation}
Assumption of a thin membrane which is isotropic and has elastic properties. 
Structures that can be actuated using inflation are concave curved surfaces. As a balloon
is inflated it pushes up against the material causing it deform. 

\subsection*{Gravity Actuation}
Auxetics can be actuated via gravity. This is effective for simple curved surfaces. 
This process involves constraining the boundary of the structure and letting gravity do 
the work. 

\subsection*{mechanical properties}
The flexibility of linkages enable the creation of a multitude of curved surfaces. Varying
deployed cells causes out of plane buckling which leads to the formation of the curved
surface. 

\subsection*{Application}
Customizable coronary stents
relocatable structures

\section*{Limitations}
Inflation and gravity actuation methods only work for limited amount of curved surfaces. 
The surfaces for which they work for positive mean curved surfaces. Does not address 
fabrication related issues. Does not address optimizing joints that are connecting the 
triangles for better deployment. 


\section*{Future Directions}
Using string as an actuation method. Testing methods to keep the structure in the deployed
state permanently. 


\section*{Citations}
\bibliographystyle{plain}
\bibliography{../../biblio.bib}
% this is how to refer to define if bib is one level above in the directory. 
Tag: curved
\cite{curved}



\end{document}