\documentclass{article}

\begin{document}
    
Title: X-Shells: A New Class of Deployable Beam Structures

Link: 

\section*{Contributions}
X-shells are a new class of deployable structures made from elastic beams. These
structures are different from the typical auxetic structure which assume rigid beams.
Aside from the elastic beams that allows for bending, these structures have joints which
allow for rotational degrees of freedom. 

The bending and torsion of beams is achieved thanks to the material properties as well 
as the joints used to hold the beams together. The joints are simply meant to add 
additional constraints onto the structure. 

Due to use of beam buckling in these types of structures there are instabilities which 
are introduced. These come in the form of unstable equilibrium which occur at the 
beginning of of the buckling process. 

X-shells are made up of elements which are deformed. 

\section*{Limitations}
Friction and gravity break symmetry in the design and manual control is required. 
In reality there are beam distortions occurring on the beam at joint locations. The 
creation of holes for the joint connections introduces points of failure in the structure. 
There is currently no formal classification as to what kinds of shapes can be constructed. 


\section*{Future Directions}
Applications for temporary structures, constrained spaces, pasta drainer. 
Compactness of these structures has additional advantages for transportation and storage. 


\section*{Citations}
\bibliographystyle{plain}
\bibliography{../../biblio.bib}
% this is how to refer to define if bib is one level above in the directory. 
Tag: xshell
\cite{xshell}



\end{document}
