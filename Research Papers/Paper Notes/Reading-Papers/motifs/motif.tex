\documentclass{article}
\usepackage{hyperref}


\begin{document}
    
Title: Bistable Auxetic Mechanical Metamaterials Inspired by Ancient Geometric Motifs

Link: \url{https://www.researchgate.net/publication/310476869_Bistable_Auxetic_Mechanical_Metamaterials_Inspired_by_Ancient_Geometric_Motifs}

\section*{Contributions}
There are auxetic structures which are monostable and some which are bistable. Monostable
structures are one which have auxetic properties but are unable to maintain their deployed
state. 

Square and triangular auxetic structures are created by creating cuts into a sheet of 
material that create a repeating pattern of unit cells connected by hinges. The hinges
connect te network of unit cells at the vertices. 

Auxetic structures derive their mechanical properties are influence in large part by the 
unit cell design rather than the chemical composition of the material from which they are
constructed. 

When the network of unit cells is pulled along one direction the hinges connecting the
unit cells rotate causing an expansion in the transverse direction, leading to the 
negative poisson's ratio. 

The paper's main focus is on triangular or square unit cells which have 3 or 4 intersecting
lines. Simple variants of these unit cells are also created by changing the angle of the 
intersecting lines. 

The principal strain in auxetic structures is on the hinges which serve as the rotational
mechanism that allows the structure to move between states. 

As I had hypothesized the theoretical deployment of auexetics uses the assumption that 
the hinges are points. In practice this is not the case because there needs to be a 
thickness which keeps the components together. This thickness also restricts the range 
of motion that the unit cells can undergo. 



\section*{Limitations}
Testing the effect of hinge thickness on mechanical properties. The paper mentions that 
they explored it but did not talk about the results in the paper. 

Origami tessellations can also be used to make auxetics but they are time consuming. 

\section*{Future Directions}
Varying the geometric properties of the unit cells to develop new designs which can be
be used to make different kinds of structures. 



\section*{Citations}
\bibliographystyle{plain}
\bibliography{../../biblio.bib}
% this is how to refer to define if bib is one level above in the directory. 
Tag: motif
\cite{motif}



\end{document}