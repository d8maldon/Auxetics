\documentclass{article}
\usepackage{hyperref}


\begin{document}
    
Title: Bistable Auxetic Mechanical Metamaterials Inspired by Ancient Geometric Motifs

Link: \url{https://www.researchgate.net/publication/310476869_Bistable_Auxetic_Mechanical_Metamaterials_Inspired_by_Ancient_Geometric_Motifs}

\section*{Contributions}
Fold patterns that make use of Resch tessellations. This pattern is similar to the pattern
that is used in auxetics that use triangles for the building block. In this paper 
\cite{tachi2013designing} they call this configuration a star tuck. Aside from this star
tuck that makes use of triangles, there are also square and hexagonal patterns. 



\section*{Limitations}
Origami tessellations can also be used to make auxetics but they are time consuming in 
terms of manufacturing and assembly. The difficulties arise due to the fact that there 
are many intricate folds that needs to be performed. 

Some folding patterns were not able to unfold from their 3D state, meaning that they were
permanently 3D. 

\section*{Future Directions}
Varying the geometric properties of the unit cells to develop new designs which can be
be used to make different kinds of structures. Optimizing surface deployability and places
where vertices do not intersect. 



\section*{Citations}
\bibliographystyle{plain}
\bibliography{../../biblio.bib}
% this is how to refer to define if bib is one level above in the directory. 
Tag: tachi2013designing
\cite{tachi2013designing}



\end{document}