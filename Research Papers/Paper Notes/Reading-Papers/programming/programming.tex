\documentclass{article}
\usepackage{hyperref}


\begin{document}
    
Title: Programming Flat-to-Synclastic Reconfiguration

Link: \url{https://www.researchgate.net/publication/330855223_Programming_Flat-to-Synclastic_Reconfiguration}

\section*{Contributions}
Defines what the mechanism that allows for bi-stability is called. The paper defines it
as snap-through buckling. When the structure is transformed from one state to another
the structure "snaps through"

The two stable states can also be defined as the two states which have an energy minimum. 
This means that the hinges are translating strain energy to other parts of the structure. 

\section*{Limitations}
Requires the use of cnc machine to create angled surfaces in the 2D case. The assumption 
that there is not strain energy on the hinges is a big assumption. 

\section*{Future Directions}
Application to larger scale structures. 



\section*{Citations}
\bibliographystyle{plain}
\bibliography{../../biblio.bib}
% this is how to refer to define if bib is one level above in the directory. 
Tag: programming
\cite{programming}



\end{document}